\lecture{2}{Mon 18 sep 2023 12:08}{}

\section{Formalization}

\begin{definition}
  Formalization is the process of giving a language a formal, mathematical description.

  Typically described in another language (meta-language.) For implementation it may be a programming language.
  Eg. Haskell
\end{definition}


\begin{definition}
  Object language: the language we're describing.
\end{definition}

\begin{definition}
  Meta-language - The language we're using to describe the object language.
\end{definition}

Things to formalize in programming languages:
\begin{itemize}
  \item Static semantics - scoping, typing.
  \item Dynamic Semantics - cost models, runtime behavior.
  \item Syntax - parsing, grammar.
\end{itemize}

There is a correspondence with formlising logics.

What logicians needed to formalize logic.

\begin{itemize}
  \item Well formedness - scoping, typing 
  \item Logical Models - proof models, truth models.
  \item Syntax - Ambiguity, grammar etc.
\end{itemize}

\section{Judgements}

\begin{definition}
  A judgement is a statement asserting a certain property for an object.
\end{definition}

Note: not necessarily a true assertion.

\begin{notation}
  judgement for property \( A \) holds for object \( s \) by writing \( sA \).
\end{notation}

\subsection{Proving Judgements}

we prove judgements by inference rules. Inference rules are crafting recipes for judgements.

\begin{notation}
  Inference rule is written as:
  \[
    \frac{J_1 \: J_2 \:\ldots \: J_n}{J}
  .\] 

  In order to prove \( J \) (conclusion), we must prove \( J_i \) (premises) where  \( i \in \{1, 2,\ldots n\}   \).
  Rules with no premises are called axioms.
\end{notation}

Continually decompose your conclusion by applying inference rules. Until you reach a base case inference rule.

When we define judgement by inference rules. we keep in mind that what is the 
smallest set that is closed under inference rules or equivalently, 
everything we can derive from the rules using finite proof trees.
This is also what it means for a definition to be inductive.

\begin{definition}
  Backward-proof:
  To prove judgement \( s A \) holds.
  \begin{itemize}
    \item Find a rule whose conclusion matches \( sA \)
    \item Preconditions of applied rule become new proof obligations.
    \item Repeat until all obligations are proven until axioms.
  \end{itemize}
\end{definition}

\subsection{Derivability}

\begin{definition}
  A rule 
  \[
    \frac{J_1 \: J_{2}\: \ldots J_n}{J}
  .\] 
  is derivable if, using the original (defining) set of rules, we can construct a proof of \( J \).
\end{definition}

\begin{definition}
 A rule is admissible if, by adding it, we don't change the meaning of the judgement i.e does not add the set of statements which satisfy the judgement.
\end{definition}

\begin{definition}
  by adding more rules, you make at least as many judgements derivable as before.
\end{definition}

\begin{notation}
 \( A \vdash B\) is equivalent to \(  \frac{A}{B} \).
\end{notation}
\subsection{Hypothetical Derivation}

\[
  \frac{A \vdash B}{C}
.\] 
Read as: If assuming \( A \) we can derive \( B \), then we can derive \( C \).

\begin{definition}
  The smallest set of rules to define every string in a language is it's minimal definition.
\end{definition}

It is also an inductive definition of the language.

\section{Rule Induction}

Informal process:

To show all strings are valid in \( M \)

\begin{itemize}
  \item prove \( P(base) \)
  \item prove premise satisfies \( P \) \( \implies \) post condition satisfies \( P \).
\end{itemize}

\begin{definition}
  Rule induction: Given a set of rules $R$, we may prove a property $P$ inductively for
all judgements that can be inferred with $R$ by showing, for each
rule of the form:

\[
  \frac{J_1 \: J_2 \: \ldots J_n }{J}
.\] 
that if \( P \) holds for premise then \( P \) holds for conclusion
\end{definition}

Note: axioms are the base cases of the induction, all other rules form
inductive cases, and the premise for each rule give rise to inductive hypothesis.
\newline
Structural induction is a special case of rule induction.



