\lecture{1}{di 12 sep 2023 19:41}{Lecture 1}
\section{Bridging the gap}

\begin{definition}
  A compiler translates from source code to a target language,
  typically machine code. 
\end{definition}
Ex. C, C++, Haskell, Rust

\begin{definition}
 An interpreter executes a program as it reads the source code 
 Examples: perl, python, javascript
\end{definition}

Note: Some languages make use of a hybrid approach. First translating
the source language to an intermediate language (abstract or virtual machine),
then interpret that. 
Examples: Java, C#

\section{Stages of a compiler}

\begin{itemize*}
The first stage of a compiler is called a lexer, which given an input string of source
code , produces a stream of tokens or lexemes, discarding irrelevant information
like whitepsace or comments.
\end{itemize*}

\begin{definition}
  The structure of lexemes expected to produce certain parse trees is called a grammar.
\end{definition}


