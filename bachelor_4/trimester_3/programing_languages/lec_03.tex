\lecture{3}{Mon 18 sep 2023 15:08}{Tut2}

1. State the formal properties of the left identity, right identity
and associativity for \( ++ \). \newline
Sol. 

\begin{itemize}
  \item \( [\;]++ys = ys \) 
  \item \( ys++[\;] = ys \) 
  \item \( (xs++ys)++zs =xs++(ys++zs)\) 
\end{itemize}

\begin{proof}

  \begin{itemize}
    \item \( [\;] ++ ys = ys \) (given)
    \item  To prove: \( ys ++ [\;] = ys \) \newline
      LHS = [\;] ++ [\;] = [\;] therefore true for base case. \newline
      Assume \( xs ++ [\;] = xs\)
      Prove true for inductive case
  \end{itemize}
  
\end{proof}
